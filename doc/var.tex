\documentclass[a4paper, english]{article}
\usepackage{graphics,eurosym,latexsym}
\usepackage{listings}
\lstset{columns=fixed,basicstyle=\ttfamily,numbers=left,numberstyle=\tiny,stepnumber=5,breaklines=true}
\usepackage{pst-all}
\usepackage{algorithmic,algorithm}
\usepackage{times}
\usepackage{babel}
\usepackage[nodayofweek]{datetime}
\usepackage[round]{natbib}
\bibliographystyle{plainnat}
\oddsidemargin=0cm
\evensidemargin=0cm
\textwidth=16cm
\textheight=23cm
\begin{document}

\title{\texttt{var} \input{version}: Compute Mean and Variance of a List of Numbers}
\author{\input{author}}
\input{date}
\date{\displaydate{tagDate}}
\maketitle

\section{Introduction} 
In daily computing it is often necessary to quickly compute the mean
and variance of a list of numbers. The program \ty{var} reads a list
of numbers either from the standard input or from a file and returns
the mean and the variance. By default it uses a two-pass algorithm to
compute the variance, which is numerically more stable but marginally
slower than its one-pass equivalent. For teaching purposes I also
implemented the one-pass algorithm, which should in all every-day
situations return the same result.

\section{Getting Started}
The program \texttt{var} was written in C on a computer running Linux.
Please contact \texttt{\input{email}} if there are any problems
with the program.
\begin{itemize}
\item Obtain the package\\
\texttt{git clone https://www.github.com/https://github.com/evolbioinf/var\ignorespaces
/var}
\item Change into the directory just downloaded
\begin{verbatim}
cd var
\end{verbatim}
and make \texttt{var}
\begin{verbatim}
make
\end{verbatim}
\item Test \texttt{var}
\begin{verbatim}
make test
\end{verbatim}
\item The executable \texttt{var} is located in the
  directory \texttt{build}. Place it into your \texttt{PATH}.
\item Make the documentation
\begin{verbatim}
make doc
\end{verbatim}
This calls the typesetting program \texttt{latex}, so please make sure
it is installed before making the documentation. The typeset manual is
located in
\begin{verbatim}
doc/var.pdf
\end{verbatim}
\item Apply to example data
\begin{verbatim}
var data/example.txt
#n	mean	var
10	0.56419	0.0887994
\end{verbatim}
The three columns contain the sample size, followed by the mean and
the variance.
\end{itemize}

\section{Change Log}
Please use
\begin{verbatim}
git log
\end{verbatim}
to list the change history of this program.

%% \bibliography{ref}
\end{document}

